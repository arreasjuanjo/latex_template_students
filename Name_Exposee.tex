% Beispiel für ein Projektexposé.
\documentclass[ a4paper,
                toc=bibliography
              ]{scrartcl}

\usepackage{iswthesis}

% Path to .bib file for BibLatex
\addbibresource{bibliography.bib}

%% Einstellungen:
\usepackage[ngerman]{babel}
\usepackage{hyperref}
\usepackage[dvipsnames]{xcolor} % Farben mit Namen ansprechen
\usepackage{pgfplots} % Plot-Engine
\pgfplotsset{compat=newest} 
\usepackage{pgfgantt} % Paket für Gantt-Diagramme

\author{Juan Jose Arrea}
\title{Prozessoptimierung für den roboterbasiertern 3D-Druck mit redundanten Achsen}
\date{\today}

\begin{document}

\maketitle
	
\section{Einleitung}

Die Relevanz von Additive Fertigung in der Produktion nimmt täglich zu. Aufgrund seiner schnellen Herstellung und vielfältige Einsatzmöglichkeiten, wird es für die Entwicklung von Prototypen oder für das Erstellen von Ersatzteilen oft bevorzugt. Die massive Skalierung des Prozesses verlangt von der Forschung große Fortschritte.

\subsection{Motivation}

Aktuell ist die Additive Fertigung nur in begrenzten Bereichen einsetzbar. 3D Drucker können aktuell Teile, die bezüglich ihrer Größe oder Komplexität herausfordernd sind, kaum produzieren. Dementsprechend werden neue Methoden geforscht, um 3D-Druck universal einsetzbar zu machen.

Die Nutzung von beweglicher Armroboter ermöglicht die Herstellung von größeren Bauteilen. Das ermöglicht 

Zum Beispiel mit Hilfe eines Armroboters.


\subsection{Zielsetzung}

Meine Aufgabe in dieser Arbeit befasst sich mit der Optimierung des roboterbasierten 3D Drucks. Dabei steht im Mittelpunkt die Aufteilung der Strecke zwischen 2 Systeme: den 6-achsigen Roboter und ein 2-achsiger Tisch. Dadurch ist eine schnellere Herstellung von Druckteilen möglich.
Zudem wird ein neues Ventil an der Düse angebracht, um die Druckqualität durch bessere Kühlung des Materials zu erhöhen. 


\subsection{Vorgehensweise}
Für die gleichzeitige Betätigung vom Tisch und Roboter mit unterschiedlichen Geschwindigkeiten wird die Externe Kommandierung eingeführt. Diese erlaubt, dass sich die Achsen vom Tisch in Abhängigkeit vom Roboter bewegen, statt diese individuell zu kommandieren. Durch einen Faktor wird die Größe der Tischstrecke in Abhängigkeit der Roboterstrecke bestimmt.
Die Funktionalität wird durch eine Simulation in Virtuos geprüft, um einen reibungslosen Ablauf am Roboter zu garantieren. Dafür wird die externe Kommandierung sowohl in die SPS der Simulation, als auch in die des Roboters implementiert.

\section{Grundlagen}

\section{Stand der Technik}

\section{Arbeitspaket- und Zeitplan}

\begin{figure*}[htbp]
    \begin{center}
        
        % Dokumentation ist unter http://mirrors.ctan.org/graphics/pgf/contrib/pgfgantt/pgfgantt.pdf einsehbar. Hier sind alle Einstellmöglichkeiten erklärt.
        \begin{ganttchart}[vgrid,
            y unit title = 0.7cm,
            y unit chart = 0.5cm,
            bar/.append style={fill=red!70},
            bar incomplete/.append style={fill=black!30},
            bar top shift =0.2,
            bar height=.5,
            group/.append style={draw=black,fill=NavyBlue!70},
            group incomplete/.append style={draw=black, fill=black!50},
            group left shift=0,
            group right shift=0,
            group height=.7,
            group peaks height = 0,
            milestone height = 0.7,
            milestone/.append style={fill=orange},
            progress=today,
            progress label text={\pgfmathprintnumber[verbatim,precision=0]{#1}\% abgeschlossen}, % Anzeige des Fortschritts auf deutsch (statt "completed")
            today = 4, % wenn today gesetzt ist wird eine Linie mit dem aktuellen Stand angezeigt.
            today label=HEUTE, % Label des aktuellen Standes/Datums deutsch (statt "TODAY")
            ]{1}{30} % Zeitraum einstellen, von wann bis wann
            
            \gantttitle{2024}{10} \gantttitle{2025}{20} \\ % wie viele Kalenderwochen in welchem Jahr
            \gantttitlelist{43,...,52}{1} \gantttitlelist{1,...,20}{1} \\ % kalenderwochen durchnummerieren.
            
            %% Hier beginnt der Inhalt der Gantt-Diagramms
            
            \ganttgroup{Paket 1}{1}{12} \\ % Task {von}{bis}  als Wochennummern
            \ganttbar{Aufgabe 1}{1}{8} \\ % Subtask
            \ganttlinkedbar{Aufgabe 2}{9}{12} \\ %Subtask verbunden mit letzter Aufgabe
            %\ganttmilestone{Milestone}{12} \\ % Meilenstein
            \ganttbar{Aufgabe 3}{13}{20} 
            
            
            \\ \ganttnewline[thick, black] % Trennlinien
        
        \end{ganttchart}
    \end{center}
    \caption{Gantt-Diagramm des geplanten Arbeitsablaufs.}\label{fig:Gantt-chart-2}
\end{figure*}


Beschreibung des Arbeitsplans in \autoref{fig:Gantt-chart-2} \ldots

% ********************************************************************
% End of contents
% ********************************************************************
\cleardoublepage
\printbibliography

\end{document}
